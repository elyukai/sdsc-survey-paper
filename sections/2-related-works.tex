\section{Related Works}\label{sec:relatedworks}

Es gibt zahlreiche Literatur im Bereich der \gls{sd}, in denen die verschiedene Methoden entweder allgemein oder speziell für einen bestimmten Anwendungsbereich beschrieben werden.

Die semantische \gls{sd} bietet Vorteile in der Verwendung von Ontologien \autocite{Novo.2020.SIitI} und findet auch im Bereich von \gls{wot} ihren Verwendungszweck \autocite{Serena.2018.SDitWoT}.
Es gibt aber auch einen Ansatz der in den Bereichen der Zuverlässigkeit und Sicherheit ihren Nutzen findet \autocite{Li..ADTCaQASDFftIoT}. Diese wird \gls{qos}-basierte \gls{sd} \autocite{Kosunalp.2020.SArlbQaIsdm} genannt und wurde unter anderem von \citeauthor{Kosunalp.2020.SArlbQaIsdm} behandelt.
Die Kontextabhängige \gls{sd} \autocite{Sukode.2015.CAFIIAS} arbeitet stattdessen mit den Informationen aller relevanten Entities.

Die \gls{sd} ist ein wichtiger Punkt in den \gls{iot}-Systemen und einen guten Einblick in dieses umfangreiche Themenfeld bietet die Arbeit \autocite{Achir.2022.SdasiIAsaat} von \citeauthor{Achir.2022.SdasiIAsaat}.

Für das \gls{wot} existiert ein sehr aktuelles und umfassendes Paper von \citeauthor{Sciullo.2022.ASotWoT}, welches auf alle Ansätze der letzten Jahrzehnte eingeht und in diversen Kategorien systematisch analysiert und vergleicht \autocite{Sciullo.2022.ASotWoT}. Auf der Webseite des \gls{w3c} finden sich verschiedene Spezifikationen und Richtlinien zum Thema \gls{wot} \autocites{w3c.wot.architecture.20200408}{w3c.wot.td.20200623}{w3c.wot.discovery.20210602}{w3c.wot.spg.20191106}.

In dieser Arbeit werden die für uns relevantesten Verfahren kurz beschrieben, was als eine Recherchearbeit für ein Projekt im Bereich der \glsxtrshort{iot}-Discovery dient.
