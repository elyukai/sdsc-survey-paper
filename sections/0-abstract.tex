\addcontentsline{toc}{section}{Abstract}
\section*{Abstract}\label{sec:abstract}

Die \gls{sd} in der Smart City beschreibt ein Verfahren, womit \glsxtrshort{iot}-Systeme in der Stadt öffentlich dem Bürger zugänglich gemacht werden.
Um dieses Verfahren zu ermöglichen, gibt es verschiedene \glsxtrshort{sd}-Ansätze, wovon drei (Semantisch, Kontextabhängig und Quality-of-Service-basiert) genauer betrachtet werden.
Diese Verfahren beschreiben aber eine abstrakte Definition und keinen vordefinierten Standard.
Somit sind diese eher ungeeignet oder mit sehr viel Aufwand verbunden, um eine \gls{sd} in der Smart City zu realisieren.
Webstandards, wie das von der W3C definierte \gls{wot}, beschreibt bereits eine genaue Architektur für eine \gls{sd}.
Mithilfe von \glspl{td} werden Metadaten mit \glspl{thing} verknüpft, welche dann von einem Service zur Verfügung gestellt werden.
Die Punkte Sicherheit und Datenschutz spielen dabei auch eine Rolle, werden aber nicht genauer betrachtet, da sie für eine öffentliche \gls{sd} nur teilweise relevant sind.
Das \gls{wot} liefert, auch wenn die Spezifikation noch nicht vollständig ist, ein genaues Verfahren, womit eine \gls{sd} in der Smart City realisiert werden kann.
Es bestehen aber immer noch Probleme bei der Definition von Ontologien und von \gls{qos}[-Aspekten] für die \gls{sd}.
