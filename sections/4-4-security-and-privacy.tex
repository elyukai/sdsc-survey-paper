\subsection{Sicherheit \& Datenschutz}\label{sec:wotsecurityprivacy}

Die Sicherheit eines \glsxtrshort{wot}-Systems kann sich auf die Sicherheit der \gls{td} selbst, oder auf die Sicherheit des \gls{thing}[s] beziehen.
Das \gls{wot} führt keine neuen Sicherheitsmechanismen ein. Es wird lediglich garantiert, dass bestehende Funktionalität und Sicherheit der Systeme beibehalten wird.
Hierbei werden verschiedene Anforderungen gestellt. Bei der Offenlegung einer \gls{td} sollte es z.\,B.\ möglich sein, bewährte Verfahren für die Sicherheit und Integrität anzuwenden.
Eine \gls{td} sollte zudem den tatsächlichen Sicherheitsstatus des von ihr beschriebenen Geräts genau wiedergeben.
Das \gls{wot} kann potenziell auf viele Protokolle angewendet werden. Für die Begrenzung des Umfangs wird sich aber hauptsächlich auf HTTP(S), CoAP(S) und MQTT(S) konzentriert.
Das \gls{wot} stellt zudem ein Framework bereit, welches eine Reihe möglicher Bedrohungen und Sicherheitsziele auflistet, die ein Anwender berücksichtigen sollten.
Des Weiteren werden Sicherheitspraktiken für die Gestaltung einer \gls{td} beschrieben, sowie ein Beispiel für eine sichere Konfiguration vorgeführt.

Mehrere Bedrohungen können die Privatsphäre von \glsxtrshort{wot}-Betreibern beeinträchtigen. Ein Betreiber kann hier z.\,B.\ eine Person sein, die ein Smart-Home besitzt,
oder eine Firma, die mehrere \glsxtrshort{wot}-\glspl{thing} in einer Fabrik betreibt.
Eine \gls{td} kann z.\,B.\ datenschutzrelevante Informationen preisgeben, wie die detaillierte Konfiguration eines einzelnen \gls{thing}[s].
Es können ebenfalls datenschutzrelevante Informationen durch Beobachtung der Kommunikation zwischen einem WoT-Endpunkt (Client, Server oder Gerät) und einer \gls{tdd} ermittelt werden.
Abhängig von der Netzwerktopologie können \glsxtrshort{wot}-Systeme Betreiberdaten zwischen dem \glsxtrshort{wot}-Konsumenten und dem \glsxtrshort{wot}-\gls{thing} über viele Zwischenknoten übertragen werden.
Wenn diese Knoten auf die Betreiberdaten zugreifen oder sie verarbeiten, kann es dazu kommen, dass ungewollt Informationen preisgeben werden.
Zusätzlich zu den oben beschriebenen Maßnahmen ist es wichtig, die Betreiber über die gesammelten Daten zu informieren und ihnen die Möglichkeit zu geben, den Grad dieser Offenlegung zu kontrollieren.
