\section{Service Discovery}\label{sec:sd}

Das \gls{iot} hat heutzutage einen großen Einfluss auf unseren Alltag.
Über eine Vielzahl von Sensoren werden dabei eine Menge an Daten aufgezeichnet, die anschließend in den unterschiedlichsten Services zur Verfügung gestellt werden.
Dabei ist das Auffinden und Auswählen des richtigen Services nicht immer ganz einfach.
Es haben sich daher unterschiedliche Arten der \gls{sd} entwickelt, die jeweils für unterschiedliche Bereiche besonders gut geeignet sind \autocite{Achir.2020.AtosdaiI}.

Im folgenden Abschnitt werden drei Arten der \gls{sd} genauer beschrieben, die am bedeutendsten sind, wenn es darum geht, Daten aus \gls{lpwan} von Städten, wie in der Einleitung beschrieben, über eine Schnittstelle für Maschinen zugänglich zu machen.
Dabei handelt es sich um die semantische \gls{sd}, die Quality-of-Service-basierte \gls{sd} und die Kontextabhängige \gls{sd}.

\subsection{Semantisch}\label{subsec:semanticsd}

Das Ziel der semantischen \gls{sd} ist es, den Service, der am besten zu den gegebenen Ansprüchen passt, mithilfe einer Ontologie zu finden.
Eine Ontologie stellt dabei die Zusammenhänge zwischen verschiedenen Begriffen oder Eigenschaften dar, die in der Suche genutzt werden können. Die Begriffe selbst werden ebenso definiert und beschrieben, um für das jeweilige Themengebiet ein einheitliches Vokabular nutzen zu können. Auf diese Weise ist es möglich, ein gemeinsames Verständnis über ein bestimmtes Gebiet zu schaffen.
Dabei können im Gegensatz zu den meisten anderen Systemen Services nicht nur funktionale, sondern auch nicht funktionale Eigenschaften -- wie Performance, Kosten oder Zuverlässigkeit -- beachtet werden.

Zum Erstellen einer semantischen Service Discovery gibt es keine festgelegte Standardisierung, die verfolgt werden muss.
Bei bisherigen Arbeiten wurde daher jeweils ein eigener Ansatz entworfen und umgesetzt. Diese Ansätze können mehr oder weniger stark voneinander abweichen.

In der Arbeit von \citeauthor{Suraci.2008.CaSSD} \autocite{Suraci.2008.CaSSD} wird eine kontextbewusste Architektur vorgestellt, die für eine Service Discovery genutzt werden kann. Die Architektur selbst ist dabei nicht abhängig von der verwendeten Technologie und kann mit vielen vorhandenen Service-Discovery-Protokollen genutzt werden.

Im Kontrast dazu handelt die Arbeit von \citeauthor{Iqbal.12320081252008.SSDuSaS} \autocite{Iqbal.12320081252008.SSDuSaS} davon, wie Services, spezifisch mit SAWSDL und SPARQL, beschrieben werden können.
Auch \citeauthor{BenMokhtar.2006.ESSDiPCE} \autocite{BenMokhtar.2006.ESSDiPCE} bauen bei ihrer Arbeit auf spezifische Sprachen auf. Sie nutzen Amigo-S für semantische Spezifikationen und S-Ariadne, eine Erweiterung von Ariadne, als Discovery-Protokoll. Ihr Ziel war es, einen effizienten Weg zum Finden von Services zu entwickeln.

In weiteren Arbeiten geht es häufig darum, wie und auf Basis welcher Grundlagen passende Services für die Nutzer ausgewählt werden können.
\citeauthor{Majithia.2004.Rbssd} \autocite{Majithia.2004.Rbssd} haben so beispielsweise ein Framework entwickelt, dass Services basierend auf ihrem Ruf findet. Zu dem Ruf eines Services gehört dabei unter anderen dessen Zuverlässigkeit und Erreichbarkeit.
\citeauthor{Jia.2017} \autocite{Jia.2017} schlagen hingegen ein zentralisiertes, mehrstufiges semantisches Service-Matching vor, das in vier Schichten abläuft, und
\citeauthor{Zhao.2017} \autocite{Zhao.2017} legen ihren Fokus auf die Beschreibung von IoT-Services und deren Ähnlichkeit zueinander. Dabei werden die Ähnlichkeiten gemessen und ein Cluster erstellt, um ähnliche Services zu sammeln.

Ein Vorteil der semantischen Service Discovery ist, dass Ontologien mit in Betracht gezogen werden können.
Aufgrund der fehlenden Standardisierung müssen Arbeiten jedoch von Grund auf eigenständig entwickelt werden.
Das kann Vorteile mit sich bringen, aber ist gleichzeitig mit einer großen Arbeitslast verbunden und nicht für alle Projekte geeignet.

\subsection{Quality-of-Service-basiert}\label{subsec:qosbasedsd}

Bei der \glsdisp{qos}{Quality-of-Service-basierten (QoS)} \gls{sd} stehen nichtfunktionale Bewertungskriterien im Vordergrund. Diese werden auf die Sicht des Anwendenden bezogen und können als Güte für deren Anforderungen verstanden werden.

Die verschiedenen \gls{qos}-Ansätze werden für die \gls{sd} in verschiedene Unterkategorien sortiert \autocite{Achir.2020.AtosdaiI}. Je nach Optimierungsziel des Ansatzes wird dieser folgenden Kategorien zugeordnet:

\begin{itemize}
    \item Sicherheitsbasiert
    \item Energieverbrauchsbasiert
    \item Netzinfrastrukturbasiert
\end{itemize}

Des Weiteren wird eine Hybrid-Kategorie gebildet, um Ansätze, die mehrere Faktoren abdecken, aufzunehmen. Eine \gls{qos}-basierte \gls{sd} verfolgt verschiedene Ziele. Die Geschwindigkeit bei der Suche soll z.\,B.\ möglichst schnell ablaufen. Algorithmen, die sich auf dieses Ziel konzentrieren, werden daher in die Netzinfrastruktur-Kategorie eingeordnet. Andere Algorithmen, die den Prozess hinsichtlich der Sicherheit der Suche z.\,B.\ mithilfe von Authentifizierungen verbessern, können für diese Arbeit vernachlässigt werden, da es sich um einen Open-Data-Standard handeln soll. Die Kategorie für den Energieverbrauch hat, zusätzlich zur Netzinfrastruktur, eine hohe Bedeutung für den Bereich dieser Arbeit. Es muss darauf geachtet werden, dass die Services möglichst wenige und kurze Berechnungen zur \gls{sd} durchführen müssen.


Die \gls{qos}-basierte \gls{sd} nutzt unterschiedliche Verarbeitungsinformationen, basiert jedoch im Grundsatz auf vorhandenen \gls{sd}-Ansätzen, wie z.\,B.\ dem semantischen Ansatz. Das Finden der Dienste basiert häufig auf den vorhandenen Ansätzen. Nachdem ein Dienst gefunden wurde, fungiert die \glsxtrshort{qos}-basierte \gls{sd} als eine Art \glsxtrshort{qos}-Filter \autocite{Kosunalp.2020.SArlbQaIsdm}. Dadurch werden Dienste, die nicht verfügbar oder nach vorhandenen Information nicht geeignet sind, eliminiert.

SARL \autocite{Kosunalp.2020.SArlbQaIsdm} ist ein Konzept, das die \glsxtrshort{qos}-basierte \gls{sd} nutzt. Es basiert auf einem P2P-Konzept, das noch keine \glsxtrshort{qos}-Faktoren abdeckt. Die einzelnen Services kennen bei diesem Ansatz nur die Art der Daten, die sie selbst bereitstellen, und Wege zu Nachbarknoten.
Es gibt verschiedene Ansätze, den Weg zum Zielknoten (Zieldienst) zu finden.

Ein nicht \glsxtrshort{qos}-basierter Ansatz ist, die Suchanfrage ohne zusätzliche Informationen zu nutzen, von Knoten zu Knoten zu schicken, bis sie beim Ziel ankommt. Dadurch entsteht ein Nachrichtenüberfluss, der die Geschwindigkeit verlangsamt und somit für Nutzende im Hinblick auf die Zuverlässigkeit wahrnehmbar ist.

Bei einem \glsxtrshort{qos}-basierten Ansatz werden zur Weiterleitung zwischen den Knoten (Diensten) verschiedene Informationen herangezogen. Es wird ein \glsxtrshort{qos}-Ranking gebildet, das die Rückgabe eines Dienstes bestimmt. Dieses basiert auf Informationen von vorherigen Anfragen oder selbstständig gesammelten Informationen. Damit verbessert der SARL-Ansatz die \gls{qos} hinsichtlich der Netzinfrastruktur.

Die Schwierigkeit bei Verbesserungen hinsichtlich der \gls{qos} besteht darin, Mittelwege zwischen aktuellen Informationen über andere Knoten und veralteten Daten zu finden. Die Netzwerklast kann beispielsweise nicht nur von zu vielen unstrukturierten Abfragen, sondern auch von zu vielen Abfragen, die den Status einzelner Knoten erfassen, erhöht werden. Dies hat eine Verletzung der \glsxtrshort{qos}-Faktoren zur Folge.

\subsection{Kontextabhängig}\label{subsec:contextawaresd}

Dieses Kapitel beschäftigt sich mit dem Thema Kontextabhängige Service Discovery
und erklärt die verschiedenen Zusammenhänge, die damit verbunden sind. Zunächst wird jedoch der einzelne Begriff Kontext beschrieben.
Dieser wird in vielen Literaturen definiert, aber in dieser Arbeit wird der englische Begriff Kontext in dem Zusammenhang mit Kontext-Awareness verwendet und sollte nicht mit direkt mit dem deutschen Begriff Kontext verwechselt werden.

Eine sehr gute und allgemeine Definition des Begriffes Kontext lautet: \textcquote{Sukode.2015.CAFIIAS}{Context is any information
that can be used to characterize the situation of an entity. An entity is a
person, place, or object that is considered relevant to the
interaction between a user and an application, including the
user and applications themselves.}

Bezogen auf unsere Arbeit bedeutet dies, dass der Kontext jede Information ist, die zur Beschreibung einer relevanten Entität verwendet werden kann. Außerdem muss der Kontext für die \gls{sd} so strukturiert sein, dass alle beteiligten Entitäten darin zusammengefasst und verwaltet werden können.

Hier nutzt die \gls{sd} einen Kontext, welcher für jeden einzelnen Anwendungsbereich neu modelliert wird. Dafür muss im Vorfeld recherchiert werden, welche Informationen die einzelnen Entitäten liefern können und für die Interaktion wichtig sind.
Ist zum Beispiel ein Sensor in der Interaktion beteiligt, so muss die Frage geklärt werden, ob der Messwert als Information ausreicht, oder ob zusätzliche Informationen wie Hersteller, Standort, oder Größe eine wichtige Rolle spielen.

Ein weiterer wichtiger Punkt ist die Art und Weise, wie die \gls{sd} alle erforderlichen Informationen abfragen kann. Häufig besitzen die Entitäten dafür eine vordefinierte Schnittstelle, welche die bei einer Abfrage die gewünschten Werte zurückliefert.
Eine weitere Möglichkeit wäre das manuelle Eintragen von Werten. Das sollte allerdings vermieden werden, damit das System autonom bleibt. Eine mögliche Ausnahme können statische Werte sein, die sich ohnehin nicht ändern.

Haben wir ein System, das solch einen Kontext berücksichtigt, dann wird von Context-Awareness gesprochen. Es nutzt bei Anfragen eines Benutzers die vorhandenen Informationen, um eine bestmögliche Antwort zu liefern.
In dem Bereich der \gls{sd} werden so gezielt Services lokalisiert und genutzt.

