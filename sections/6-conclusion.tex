\section{Fazit}\label{sec:conclusion}

Die Recherche im Rahmen eines Ansatzes für eine Open-Data-Schnittstelle in Smart Cities hat gezeigt, dass die \gls{wot} vom \gls{w3c} einen guten Ansatz bieten, um nach diesem Schema eine entsprechende Schnittstelle zu implementieren. Die Standards des \gls{w3c} bieten Standardisierungen, die z.\,B.\ das Beschreiben von Sensoren als auch die \gls{sd} abdecken. Zudem handelt es sich im Vergleich zu den untersuchten \gls{sd}-Aufbauten um aktuelle und sehr umfangreiche Beschreibungen des Standards. An einigen Stellen ist der Standard jedoch noch nicht vollständig ausgearbeitet.

Die Nutzung der diskutierten \gls{sd} aus den entsprechenden Papern hat sich als nicht sinnvoll herausgestellt, da diese oft für spezifische Anwendungsfälle entwickelt worden sind und somit keine Standardisierung vorhanden ist. Daher wird empfohlen, sich bei der Implementierung einer Schnittstelle an dem \glsxtrshort{wot}-Standard zu orientieren, der zusätzlich den Vorteil mitbringt, dass er nicht nur die \gls{sd} beschreibt.
